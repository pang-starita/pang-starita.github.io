% Options for packages loaded elsewhere
\PassOptionsToPackage{unicode}{hyperref}
\PassOptionsToPackage{hyphens}{url}
%
\documentclass[12pt, a4paper
]{article}
\usepackage{lmodern}
\usepackage{amssymb,amsmath}
\usepackage{ifxetex,ifluatex}
\usepackage{indentfirst}
\usepackage[a4paper, margin=2cm]{geometry}
\ifnum 0\ifxetex 1\fi\ifluatex 1\fi=0 % if pdftex
  \usepackage[T1]{fontenc}
  \usepackage[utf8]{inputenc}
  \usepackage{textcomp} % provide euro and other symbols
\else % if luatex or xetex
  \usepackage{unicode-math}
  \defaultfontfeatures{Scale=MatchLowercase}
  \defaultfontfeatures[\rmfamily]{Ligatures=TeX,Scale=1}
\fi
% Use upquote if available, for straight quotes in verbatim environments
\IfFileExists{upquote.sty}{\usepackage{upquote}}{}
\IfFileExists{microtype.sty}{% use microtype if available
  \usepackage[]{microtype}
  \UseMicrotypeSet[protrusion]{basicmath} % disable protrusion for tt fonts
}{}
\makeatletter
\@ifundefined{KOMAClassName}{% if non-KOMA class
  \IfFileExists{parskip.sty}{%
    \usepackage{parskip}
  }{% else
    \setlength{\parindent}{0pt}
    \setlength{\parskip}{6pt plus 2pt minus 1pt}}
}{% if KOMA class
  \KOMAoptions{parskip=half}}
\makeatother
\usepackage{xcolor}
\IfFileExists{xurl.sty}{\usepackage{xurl}}{} % add URL line breaks if available
\IfFileExists{bookmark.sty}{\usepackage{bookmark}}{\usepackage{hyperref}}
\hypersetup{
  pdftitle={O Que É o REST?},
  pdfauthor={Paulino Ng},
  hidelinks,
  pdfcreator={LaTeX via pandoc}}
\urlstyle{same} % disable monospaced font for URLs
\usepackage{longtable,booktabs}
% Correct order of tables after \paragraph or \subparagraph
\usepackage{etoolbox}
\makeatletter
\patchcmd\longtable{\par}{\if@noskipsec\mbox{}\fi\par}{}{}
\makeatother
% Allow footnotes in longtable head/foot
\IfFileExists{footnotehyper.sty}{\usepackage{footnotehyper}}{\usepackage{footnote}}
\makesavenoteenv{longtable}
\setlength{\emergencystretch}{3em} % prevent overfull lines
\providecommand{\tightlist}{%
  \setlength{\itemsep}{0pt}\setlength{\parskip}{0pt}}
\setcounter{secnumdepth}{-\maxdimen} % remove section numbering

\title{O Que É o REST?}
\author{Paulino Ng}
\date{2019-09-04}

\begin{document}
\maketitle

\hypertarget{aprenda-sobre-como-projetar-serviuxe7os-da-web-usando-o-paradigma-rest}{%
\subsection{Aprenda sobre como projetar serviços da Web usando o
paradigma
REST}\label{aprenda-sobre-como-projetar-serviuxe7os-da-web-usando-o-paradigma-rest}}

\textbf{REST}, ou \emph{REpresentational State Transfer}, é um estilo de
arquitetura para fornecer padrões entre sistemas de computador na Web,
facilitando a comunicação entre os sistemas. Os sistemas compatíveis com
\textbf{REST}, geralmente chamados de sistemas \emph{RESTful}, são
caracterizados pela forma como são \emph{sem estado} e isolam as
preocupações do cliente e do servidor. Explicaremos o que esses termos
significam e por que eles são características benéficas para serviços na
Web.

\hypertarget{separauxe7uxe3o-do-cliente-e-do-servidor}{%
\subsection{Separação do Cliente e do Servidor}\label{separauxe7uxe3o-do-cliente-e-do-servidor}}

No estilo arquitetural \textbf{REST}, a implementação do cliente e a
implementação do servidor podem ser feitas independentemente, sem que
cada um tenha conhecimento do outro. Isso significa que o código no lado
do cliente pode ser alterado a qualquer momento sem afetar a operação do
servidor e o código no lado do servidor pode ser alterado sem afetar a
operação do cliente.

Desde que cada lado saiba qual formato de mensagens enviar para o outro,
eles podem ser mantidos modulares e separados. Separando as preocupações
da interface do usuário das preocupações com armazenamento de dados,
melhoramos a flexibilidade da interface entre as plataformas e
aprimoramos a escalabilidade, simplificando os componentes do servidor.
Além disso, a separação permite que cada componente evolua
independentemente.

Ao usar uma interface \textbf{REST}, diferentes clientes chegam nos
mesmos pontos extremos \textbf{REST}, executam as mesmas ações e recebem
as mesmas respostas.

\hypertarget{ausuxeancia-de-estado}{%
\subsection{Ausência de Estado}\label{ausuxeancia-de-estado}}

Sistemas que seguem o paradigma \textbf{REST} são sem estado, o que
significa que o servidor não precisa saber nada sobre o estado do
cliente e vice-versa. Dessa forma, tanto o servidor quanto o cliente
podem entender qualquer mensagem recebida, mesmo sem ver as mensagens
anteriores. Esta restrição de ausência de estado é imposta através do
uso de recursos, em vez de comandos. Recursos são os substantivos da Web
- eles descrevem qualquer objeto, documento ou coisa que você pode
precisar armazenar ou enviar para outros serviços.

Como os sistemas \textbf{REST} interagem através de operações padrões
nos recursos, eles não dependem da implementação de interfaces.

Essas restrições ajudam os aplicativos \textbf{RESTful} a obterem
confiabilidade, desempenho rápido e escalabilidade, como componentes que
podem ser gerenciados, atualizados e reutilizados sem afetar o sistema
como um todo, mesmo durante a operação do sistema.

Agora, vamos explorar como a comunicação entre o cliente e o servidor
realmente acontece quando estamos implementando uma interface
\textbf{RESTful}.

\hypertarget{comunicauxe7uxe3o-entre-o-cliente-e-o-servidor}{%
\subsection{Comunicação entre o Cliente e o Servidor}\label{comunicauxe7uxe3o-entre-o-cliente-e-o-servidor}}

Na arquitetura \textbf{REST}, os clientes enviam solicitações para obter
ou modificar recursos e os servidores enviam respostas a essas
solicitações. Vamos dar uma olhada nas formas padrões de fazer
solicitações e enviar respostas.

\hypertarget{envio-de-requisiuxe7uxf5es}{%
\subsubsection{Envio de Requisições}\label{envio-de-requisiuxe7uxf5es}}

O \textbf{REST} exige que um cliente faça uma solicitação ao servidor
para recuperar ou modificar dados no servidor. Uma solicitação
geralmente consiste em:

\begin{itemize}
\tightlist
\item
  um verbo HTTP, que define que tipo de operação realizar
\item
  um cabeçalho, que permite ao cliente passar informações sobre a
  solicitação
\item
  um caminho para um recurso
\item
  um corpo de mensagem opcional contendo dados
\end{itemize}

\hypertarget{verbos-do-http}{%
\subsubsection{Verbos do HTTP}\label{verbos-do-http}}

Existem 4 verbos básicos de HTTP que usamos em pedidos para interagir
com recursos em um sistema \textbf{REST}:

\begin{itemize}
\tightlist
\item
  \textbf{GET} - recuperar um recurso específico (por id) ou uma coleção
  de recursos
\item
  \textbf{POST} - criar um novo recurso
\item
  \textbf{PUT} - atualizar um recurso específico (por id)
\item
  \textbf{DELETE} - remover um recurso específico (por id)
\end{itemize}

\hypertarget{paruxe2metros-no-header-e-no-accept}{%
\subsubsection{Parâmetros no Header e no
Accept}\label{paruxe2metros-no-header-e-no-accept}}

No cabeçalho da solicitação, o cliente envia o tipo de conteúdo que pode
receber do servidor. Isso é chamado de campo de \texttt{Accept} e
garante que o servidor não envie dados que não possam ser compreendidos
ou processados pelo cliente. As opções para \texttt{content-type} são
\emph{MIME Types} (\emph{Multipurpose Internet Mail Extensions}), sobre
as quais você pode se aprofundar em
\href{https://developer.mozilla.org/en-US/docs/Web/HTTP/Basics_of_HTTP/MIME_types}{MDN
Web Docs}. Os \emph{MIME Types}, usados para especificar os tipos de
conteúdo no campo \texttt{Accept}, consistem em um tipo e um subtipo.
Eles são separados por uma barra inclinada (/).

Por exemplo, um arquivo de texto contendo HTML seria especificado com o
tipo \texttt{text/html}. Se esse arquivo de texto contivesse \emph{CSS},
ele seria especificado como \texttt{text/css}. Um arquivo de texto
genérico seria denotado como \texttt{text/plain}. Este valor padrão,
\texttt{text/plain}, não é um faz-tudo, substituto para qualquer coisa. Se um cliente está
esperando \texttt{text/css} e recebe \texttt{text/plain}, ele não será
capaz de reconhecer o conteúdo e renderizar adequadamente uma página HTML.

Outros tipos e subtipos comumente usados são:

\begin{itemize}
\tightlist
\item
  \textbf{imagem} --- \texttt{image/png}, \texttt{image/jpeg},
  \texttt{image/gif}
\item
  \textbf{áudio} --- \texttt{audio/wav}, \texttt{image/mpeg}
\item
  \textbf{vídeo} --- \texttt{video/mp4}, \texttt{video/ogg}
\item
  \textbf{aplicação} --- \texttt{application/json},
  \texttt{application/pdf}, \texttt{application/xml},\\
  \texttt{application/octet-stream}
\end{itemize}

Por exemplo, um cliente acessando um recurso com id 23 em um recurso
\texttt{articles} em um servidor pode enviar um pedido \textbf{GET}
assim:

\begin{verbatim}
GET /articles/23
Accept: text/html, application/xhtml
\end{verbatim}

O campo de cabeçalho \texttt{Accept} neste caso está dizendo que o
cliente aceitará o conteúdo em \texttt{text/html} ou
\texttt{application/xhtml}.

\hypertarget{paths}{%
\subsubsection{PATHs}\label{paths}}

As solicitações devem conter um caminho (\emph{path}) para um recurso no
qual a operação deve ser executada. Nas APIs \textbf{RESTful}, os
caminhos devem ser projetados para ajudar o cliente a saber o que está
acontecendo.

Convencionalmente, a primeira parte do caminho deve ser a forma plural
do recurso. Isso mantém os caminhos aninhados simples de ler e fáceis de
entender.

Um caminho como \texttt{fashionboutique.com/customers/223/orders/12} é
claro no que ele indica, mesmo que você nunca tenha visto esse caminho
específico antes, porque é hierárquico e descritivo. Podemos ver que
estamos acessando o pedido com id 12 para o cliente com id 223. Os
caminhos devem conter as informações necessárias para localizar um
recurso com o grau de especificidade necessário. Ao se referir a uma
lista ou coleção de recursos, não é necessário adicionar um ID a uma
solicitação \texttt{POST}. Por exemplo, para um caminho \texttt{fashionboutique.com/customers}
não precisa de um identificador extra, pois o servidor
gerará um id para o novo objeto.

Se estamos tentando acessar um único recurso, precisaríamos acrescentar
um ID ao caminho. Por exemplo:
\texttt{GET\ fashionboutique.com/customers/:id} - recupera o item no
recurso \texttt{customers} com o id especificado.
\texttt{DELETE\ fashionboutique.com/customers/:id} - exclui o item no
recurso \texttt{customers} com o id especificado.

\hypertarget{envio-de-respostas}{%
\subsubsection{Envio de Respostas}\label{envio-de-respostas}}

\hypertarget{tipos-de-conteuxfado}{%
\subsubsection{Tipos de Conteúdo}\label{tipos-de-conteuxfado}}

Nos casos em que o servidor está enviando dados para o
cliente, o servidor deve incluir um \texttt{Content-Type} no cabeçalho
da resposta. Esse campo de cabeçalho \texttt{Content-Type} alerta o
cliente para o tipo de dados que está enviando no corpo da resposta.
Esses tipos de conteúdo são tipos de MIME, assim como estão no campo
\texttt{Accept} do cabeçalho da solicitação. O \texttt{Content-Type} que
o servidor envia de volta na resposta deve ser de uma das opções que o
cliente especificou no campo \texttt{Accept} da solicitação.

Por exemplo, quando um cliente está acessando um recurso com o id 23 em
um recurso \texttt{articles} com essa requisição \texttt{GET}:

\begin{verbatim}
GET /articles/23 HTTP/1.1
Accept: text/html, application/xhtml
\end{verbatim}

O servidor pode enviar de volta o conteúdo com o cabeçalho de resposta :

\begin{verbatim}
HTTP/1.1 200 (OK)
Content-Type: text/html
\end{verbatim}

Isso significa que o conteúdo solicitado está retornando no corpo da
resposta com um \texttt{Content-Type} de \texttt{text/html}, que o
cliente disse que aceitaria.

\hypertarget{cuxf3digos-de-resposta}{%
\subsubsection{Códigos de Resposta}\label{cuxf3digos-de-resposta}}

As respostas do servidor contém códigos de \emph{status} para alertar o
cliente com informações sobre o sucesso da operação. Como desenvolvedor,
você não precisa conhecer todos os códigos de status (há muitos deles),
mas você deve conhecer os mais comuns e como eles são usados:

\begin{longtable}[]{@{}ll@{}}
\toprule
\begin{minipage}[b]{0.47\columnwidth}\raggedright
Código de Status\strut
\end{minipage} & \begin{minipage}[b]{0.47\columnwidth}\raggedright
Significado\strut
\end{minipage}\tabularnewline
\midrule
\endhead
\begin{minipage}[t]{0.47\columnwidth}\raggedright
\texttt{200\ (OK)}\strut
\end{minipage} & \begin{minipage}[t]{0.47\columnwidth}\raggedright
Esta é a resposta padrão para requisições de HTTP bem-sucedidas.\strut
\end{minipage}\tabularnewline
\begin{minipage}[t]{0.47\columnwidth}\raggedright
\texttt{201\ (CREATED)}\strut
\end{minipage} & \begin{minipage}[t]{0.47\columnwidth}\raggedright
Esta é a resposta padrão para uma requisição de HTTP que resultou em um
item sendo criado com sucesso.\strut
\end{minipage}\tabularnewline
\begin{minipage}[t]{0.47\columnwidth}\raggedright
\texttt{204\ (NO\ CONTENT)}\strut
\end{minipage} & \begin{minipage}[t]{0.47\columnwidth}\raggedright
Esta é a resposta padrão para requisições de HTTP bem-sucedidas, em que
nada está sendo retornado no corpo da resposta.\strut
\end{minipage}\tabularnewline
\begin{minipage}[t]{0.47\columnwidth}\raggedright
\texttt{400\ (BAD\ REQUEST)}\strut
\end{minipage} & \begin{minipage}[t]{0.47\columnwidth}\raggedright
A requisição não pode ser processada devido à sintaxe de requisição
incorreta, tamanho excessivo ou outro erro do cliente.\strut
\end{minipage}\tabularnewline
\begin{minipage}[t]{0.47\columnwidth}\raggedright
\texttt{403\ (FORBIDDEN)}\strut
\end{minipage} & \begin{minipage}[t]{0.47\columnwidth}\raggedright
O cliente não tem permissão para acessar este recurso.\strut
\end{minipage}\tabularnewline
\begin{minipage}[t]{0.47\columnwidth}\raggedright
\texttt{404\ (NOT\ FOUND)}\strut
\end{minipage} & \begin{minipage}[t]{0.47\columnwidth}\raggedright
O recurso não pôde ser encontrado neste momento. É possível que tenha
sido apagado ou não exista.\strut
\end{minipage}\tabularnewline
\begin{minipage}[t]{0.47\columnwidth}\raggedright
\texttt{500\ (INTERNAL\ SERVER\ ERROR)}\strut
\end{minipage} & \begin{minipage}[t]{0.47\columnwidth}\raggedright
A resposta genérica para uma falha inesperada se não houver mais
informações específicas disponíveis.\strut
\end{minipage}\tabularnewline
\bottomrule
\end{longtable}

Para cada verbo HTTP, há códigos de \emph{status} esperados que um
servidor deve retornar após o sucesso:

\begin{itemize}
\tightlist
\item
  \textbf{GET} - retornar \texttt{200\ (OK)}
\item
  \textbf{POST} - retornar \texttt{201\ (CREATED)}
\item
  \textbf{PUT} - retornar \texttt{200\ (OK)}
\item
  \textbf{DELETE} - retornar \texttt{204\ (NO\ CONTENT)} Se a operação
  falhar, retorne o código de status mais específico possível
  correspondente ao problema encontrado.
\end{itemize}

\hypertarget{exemplos-de-requisiuxe7uxf5es-e-respostas}{%
\subsubsection{Exemplos de Requisições e
Respostas}\label{exemplos-de-requisiuxe7uxf5es-e-respostas}}

Digamos que temos um aplicativo que permite visualizar, criar, editar e
excluir clientes e pedidos de uma pequena loja de roupas hospedada no
\texttt{fashionboutique.com}. Poderíamos criar uma API HTTP que
permitisse a um cliente executar estas funções:

Se quiséssemos visualizar todos os clientes (\emph{customers}), a
requisição ficaria assim:

\begin{verbatim}
GET http://fashionboutique.com/customers
Accept: application/json
\end{verbatim}

Um cabeçalho de resposta possível seria parecido com:

\begin{verbatim}
Status Code: 200 (OK)
Content-Type: application/json
\end{verbatim}

seguido dos dados requisitados dos clientes no formato
\texttt{application/json}.

Crie um novo cliente postando os dados:

\begin{verbatim}
POST http://fashionboutique.com/customers
Body:
{
  "customer": {
    "name" = "Walter Carvalho"
    "email" = "walter.carvalho@loucademia.org"
  }
}
\end{verbatim}

O servidor, em seguida, gera um ID para esse objeto e retorna-o de volta
para o cliente, com um cabeçalho do tipo:

\begin{verbatim}
201 (CREATED)
Content-Type: application/json
\end{verbatim}

Para visualizar um único cliente usamos o \texttt{GET} especificando o
id do cliente:

\begin{verbatim}
GET http://fashionboutique.com/customers/123
Accept: application/json
\end{verbatim}

Um cabeçalho de resposta possível seria semelhante a:

\begin{verbatim}
Status Code: 200 (OK)
Content-Type: application/json
\end{verbatim}

seguido pelos dados para o recurso do \texttt{customer} com id 23 no
formato \texttt{application/json}.

\end{document}
